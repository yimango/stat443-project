\documentclass[aspectratio=169]{beamer}
\usepackage[utf8]{inputenc}
\usepackage[T1]{fontenc}
\usepackage{amsmath,amssymb}
\usepackage{graphicx}
\usepackage{booktabs}
\usepackage{natbib}
\usepackage{xcolor}

% Beamer theme
\usetheme{Madrid}
\usecolortheme{default}

% Title information
\title{SP500 Time-Series Forecasting}
\subtitle{A Box-Jenkins and GARCH Approach}
\author{Justin Wang and Remi Liu}
\institute{STAT 443\\2025}
\date{November 2025}

\begin{document}

% Title slide
\begin{frame}
\titlepage
\end{frame}

% Outline
\begin{frame}{Outline}
\tableofcontents
\end{frame}

% ============================================
% INTRODUCTION
% ============================================
\section{Introduction}

\begin{frame}{Problem Statement}
\begin{itemize}
    \item \textbf{Objective}: Forecast SP500 returns and volatility using time-series methods
    \item \textbf{Challenge}: Financial time series exhibit:
    \begin{itemize}
        \item Non-stationarity
        \item Volatility clustering
        \item Complex dependencies
    \end{itemize}
    \item \textbf{Approach}: 
    \begin{itemize}
        \item Box-Jenkins methodology (AR, MA, ARIMA) for returns
        \item GARCH/ARCH models for volatility
    \end{itemize}
\end{itemize}
\end{frame}

\begin{frame}{Why It Matters}
\begin{columns}
\column{0.5\textwidth}
\textbf{Investment Management}
\begin{itemize}
    \item Portfolio allocation
    \item Market timing
    \item Risk-return tradeoffs
\end{itemize}

\column{0.5\textwidth}
\textbf{Risk Management}
\begin{itemize}
    \item Value-at-Risk (VaR)
    \item Position sizing
    \item Hedging strategies
\end{itemize}
\end{columns}

\vspace{0.5cm}
\textbf{Derivatives Pricing}
\begin{itemize}
    \item Option pricing requires volatility forecasts
    \item Black-Scholes and other models
\end{itemize}
\end{frame}

% ============================================
% DATA
% ============================================
\section{Data}

\begin{frame}{Data Sources}
\begin{itemize}
    \item \textbf{Period}: October 1, 2015 to October 30, 2025
    \item \textbf{Observations}: 2,631 daily observations
    \item \textbf{Primary Data}:
    \begin{itemize}
        \item SP500 ETF (SPY): Daily closing prices
        \item VIX Index: Market volatility expectations
        \item 10-Year Treasury Yield: Risk-free rate proxy
        \item High-Yield Credit Spread: Credit risk indicator
    \end{itemize}
    \item \textbf{Target Variable}: Daily log returns
    \[
    r_t = \log(P_t) - \log(P_{t-1})
    \]
\end{itemize}
\end{frame}

\begin{frame}{Data Characteristics}
\begin{columns}
\column{0.5\textwidth}
\textbf{Descriptive Statistics}
\begin{itemize}
    \item Mean return: 0.04\% daily
    \item Std dev: 1\% daily
    \item Distribution: Fat-tailed
    \item Volatility clustering: Present
\end{itemize}

\column{0.5\textwidth}
\textbf{Stationarity Tests}
\begin{itemize}
    \item ADF test: p $<$ 0.05
    \item KPSS test: p $>$ 0.05
    \item \textbf{Result}: Returns are stationary (d=0)
    \item No differencing required
\end{itemize}
\end{columns}
\end{frame}

\begin{frame}{Feature Engineering}
\begin{itemize}
    \item Created \textbf{36 potential predictors} from raw market data
    \item Categories:
    \begin{itemize}
        \item Lagged returns (R\_lag1, R\_lag2, R\_lag5)
        \item Realized volatility (5-day, 20-day)
        \item Technical indicators (RSI, moving averages)
        \item Cross-asset features (VIX ratios, yield curves)
        \item Interaction terms
    \end{itemize}
    \item \textbf{Variable Selection}: Elastic Net regularization
    \begin{itemize}
        \item Selected \textbf{8 predictors} (22\% selection rate)
        \item Stability threshold: $\geq$ 0.6
    \end{itemize}
\end{itemize}
\end{frame}

% ============================================
% METHODOLOGY
% ============================================
\section{Methodology}

\begin{frame}{Box-Jenkins Methodology}
\textbf{Three-Step Procedure}:
\begin{enumerate}
    \item \textbf{Identification}
    \begin{itemize}
        \item ACF/PACF analysis
        \item Stationarity testing
    \end{itemize}
    \item \textbf{Estimation}
    \begin{itemize}
        \item Grid search: AR($p$), MA($q$), ARIMA($p,0,q$)
        \item Selection criteria: AIC, BIC
    \end{itemize}
    \item \textbf{Diagnostic Checking}
    \begin{itemize}
        \item Ljung-Box test (residual autocorrelation)
        \item Jarque-Bera test (normality)
        \item Residual plots
    \end{itemize}
\end{enumerate}
\end{frame}

\begin{frame}{Volatility Models}
\begin{columns}
\column{0.5\textwidth}
\textbf{ARCH(q) Model}
\[
\sigma_t^2 = \omega + \sum_{i=1}^{q} \alpha_i \varepsilon_{t-i}^2
\]
\begin{itemize}
    \item Models volatility using past squared errors
    \item Simpler, fewer parameters
\end{itemize}

\column{0.5\textwidth}
\textbf{GARCH(p,q) Model}
\[
\sigma_t^2 = \omega + \sum_{i=1}^{q} \alpha_i \varepsilon_{t-i}^2 + \sum_{j=1}^{p} \beta_j \sigma_{t-j}^2
\]
\begin{itemize}
    \item Includes lagged variance terms
    \item Captures volatility persistence
    \item GARCH(1,1) is standard in finance
\end{itemize}
\end{columns}
\end{frame}

\begin{frame}{Backtesting Procedure}
\begin{itemize}
    \item \textbf{Training Set}: 80\% (2015-10-01 to 2023-10-23)
    \item \textbf{Test Set}: 20\% (2023-10-24 to 2025-10-30)
    \item \textbf{Method}: Rolling-origin expanding window
    \item \textbf{Forecast Horizon}: 1-step-ahead
    \item \textbf{Test Folds}: 528 (one per test observation)
    \item \textbf{No Future Leakage}: Each forecast uses only past data
\end{itemize}

\vspace{0.3cm}
\textbf{Evaluation Metrics}:
\begin{itemize}
    \item RMSE, MAE, MAPE
    \item Directional accuracy
    \item Diebold-Mariano test
\end{itemize}
\end{frame}

% ============================================
% RESULTS
% ============================================
\section{Results}

\begin{frame}{Model Selection: Return Forecasting}
\begin{table}
\centering
\small
\begin{tabular}{lcccc}
\toprule
Model & RMSE & MAE & MAPE (\%) & Dir. Acc. (\%) \\
\midrule
\textbf{ARIMA(2,0,2)} & \textbf{0.0100} & 0.00655 & 6439 & 49.6 \\
AR(8) & 0.0101 & 0.00653 & 6969 & 51.1 \\
MA(2) & 0.0101 & 0.00640 & 2838 & \textbf{53.4} \\
\bottomrule
\end{tabular}
\end{table}

\vspace{0.3cm}
\textbf{Selected: ARIMA(2,0,2)}
\begin{itemize}
    \item Lowest RMSE (best overall accuracy)
    \item AIC: -16,287.1, BIC: -16,251.85
    \item Combines AR and MA components
\end{itemize}
\end{frame}

\begin{frame}{Model Selection: Volatility Forecasting}
\begin{table}
\centering
\small
\begin{tabular}{lcccc}
\toprule
Model & Order & AIC & BIC & Mean Vol. (\%) \\
\midrule
\textbf{GARCH} & \textbf{(1,1)} & \textbf{-6.65} & \textbf{-6.64} & \textbf{0.931} \\
ARCH & (1) & -6.14 & -6.14 & 1.152 \\
\bottomrule
\end{tabular}
\end{table}

\vspace{0.3cm}
\textbf{Selected: GARCH(1,1)}
\begin{itemize}
    \item Lower BIC (-6.64 vs -6.14)
    \item Time-varying volatility (0.855\% to 0.991\%)
    \item ARCH produces constant volatility (unrealistic)
\end{itemize}
\end{frame}

\begin{frame}{Model Diagnostics}
\begin{columns}
\column{0.5\textwidth}
\textbf{ARIMA(2,0,2)}
\begin{itemize}
    \item Ljung-Box: Some residual autocorrelation (acceptable)
    \item Jarque-Bera: Non-normal residuals (expected)
    \item Residual ACF: Mostly within bands
\end{itemize}

\column{0.5\textwidth}
\textbf{GARCH(1,1)}
\begin{itemize}
    \item Residuals: White noise
    \item Squared residuals: No ARCH effects
    \item Volatility clustering: Captured
\end{itemize}
\end{columns}

\vspace{0.5cm}
\textbf{Diebold-Mariano Tests}:
\begin{itemize}
    \item No statistically significant differences between AR, MA, ARIMA
    \item All models have similar predictive power
\end{itemize}
\end{frame}

\begin{frame}{Return Forecasts}
\textbf{21-Day Forecast (Oct 31 - Nov 28, 2025)}:
\begin{itemize}
    \item \textbf{Cumulative Return}: +1.12\%
    \item \textbf{Annualized}: $\sim$13.4\%
    \item \textbf{Directional}: 81\% positive days (17 up, 4 down)
    \item \textbf{Mean Daily Return}: +0.053\%
    \item \textbf{Prediction Intervals}: 95\% width averages 4.36\%
\end{itemize}

\vspace{0.3cm}
\textbf{Interpretation}:
\begin{itemize}
    \item Bullish short-term outlook
    \item Moderate volatility expected
    \item Wide prediction intervals reflect uncertainty
\end{itemize}
\end{frame}

\begin{frame}{Volatility Forecasts}
\textbf{GARCH(1,1) Forecast}:
\begin{itemize}
    \item \textbf{Mean Volatility}: 0.931\% (vs historical: 0.843\%)
    \item \textbf{Range}: 0.855\% to 0.991\% (time-varying)
    \item \textbf{Change}: +10.4\% increase from historical mean
    \item \textbf{Trend}: Gradually increasing over 21 days
\end{itemize}

\vspace{0.3cm}
\textbf{Implications}:
\begin{itemize}
    \item Increasing market risk ahead
    \item Consider reducing position sizes
    \item Increase hedging activities
    \item Adjust VaR calculations upward
\end{itemize}
\end{frame}

\begin{frame}{Forecast Visualization}
\begin{center}
\includegraphics[width=0.85\textwidth]{outputs/sp500_forecast.png}
\end{center}
\end{frame}

\begin{frame}{Volatility Forecast Visualization}
\begin{center}
\includegraphics[width=0.85\textwidth]{outputs/volatility_forecast.png}
\end{center}
\end{frame}

% ============================================
% CONCLUSIONS
% ============================================
\section{Conclusions}

\begin{frame}{Key Findings}
\begin{enumerate}
    \item \textbf{Best Return Model}: ARIMA(2,0,2)
    \begin{itemize}
        \item RMSE: 0.0100 (best accuracy)
        \item All models show similar performance
    \end{itemize}
    
    \item \textbf{Best Volatility Model}: GARCH(1,1)
    \begin{itemize}
        \item BIC: -6.64 (clearly superior to ARCH)
        \item Produces realistic time-varying forecasts
    \end{itemize}
    
    \item \textbf{Forecast Insights}:
    \begin{itemize}
        \item Bullish short-term outlook (+1.12\% over 21 days)
        \item Increasing volatility (+10.4\% from historical)
    \end{itemize}
\end{enumerate}
\end{frame}

\begin{frame}{Practical Implications}
\begin{columns}
\column{0.5\textwidth}
\textbf{Investment Strategy}
\begin{itemize}
    \item Maintain equity exposure
    \item 81\% positive days forecast
    \item But account for uncertainty
\end{itemize}

\column{0.5\textwidth}
\textbf{Risk Management}
\begin{itemize}
    \item Reduce position sizes
    \item Increase hedging
    \item Adjust VaR upward
\end{itemize}
\end{columns}

\vspace{0.5cm}
\textbf{Caveat}: Forecasts are probabilistic, not deterministic. Actual outcomes may differ, especially during unexpected market events.
\end{frame}

\begin{frame}{Limitations and Future Work}
\textbf{Limitations}:
\begin{itemize}
    \item Stationarity assumption may not hold during regime changes
    \item Linear models may miss non-linear dependencies
    \item Short-term focus (1-step-ahead forecasts)
    \item No exogenous variables in pure time-series approach
\end{itemize}

\vspace{0.3cm}
\textbf{Future Improvements}:
\begin{itemize}
    \item ARIMAX models with external predictors
    \item GARCH extensions (GJR-GARCH, EGARCH)
    \item Regime-switching models
    \item Machine learning comparisons (LSTM, XGBoost)
\end{itemize}
\end{frame}

\begin{frame}{Summary}
\begin{itemize}
    \item Successfully applied Box-Jenkins and GARCH methodologies to SP500 forecasting
    \item Selected ARIMA(2,0,2) for returns and GARCH(1,1) for volatility
    \item Generated practical forecasts with prediction intervals
    \item Provided actionable insights for investment and risk management
\end{itemize}

\end{frame}

\end{document}


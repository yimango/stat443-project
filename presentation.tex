\documentclass[aspectratio=169]{beamer}
\usepackage[utf8]{inputenc}
\usepackage[T1]{fontenc}
\usepackage{amsmath,amssymb}
\usepackage{graphicx}
\usepackage{booktabs}
\usepackage{natbib}
\usepackage{xcolor}

% Beamer theme
\usetheme{Madrid}
\usecolortheme{default}

% Title information
\title{SP500 Time-Series Forecasting}
\subtitle{A Box-Jenkins and GARCH Approach}
\author{Justin Wang and Dingfei Liu}
\institute{STAT 443\\2025}
\date{November 2025}

\begin{document}

% Title slide
\begin{frame}
\titlepage
\end{frame}

% ============================================
% PROBLEM
% ============================================
\begin{frame}{Problem \& Motivation}
\begin{itemize}
    \item \textbf{Objective}: Forecast SP500 returns and volatility
    \item \textbf{Challenges}: Non-stationarity, volatility clustering, complex dependencies
    \item \textbf{Approach}: 
    \begin{itemize}
        \item Box-Jenkins (AR, MA, ARIMA) for returns
        \item GARCH/ARCH for volatility
    \end{itemize}
    \item \textbf{Applications}: Investment management, risk management, derivatives pricing
\end{itemize}
\end{frame}

% ============================================
% DATA
% ============================================
\begin{frame}{Data}
\begin{itemize}
    \item \textbf{Period}: October 1, 2015 to October 30, 2025
    \item \textbf{Observations}: Over 2,000 daily observations
    \item \textbf{Target}: Daily log returns
    \item \textbf{Key Features}: 
    \begin{itemize}
        \item Stationary (d=0) - no differencing required
        \item Fat-tailed distribution
        \item Volatility clustering present
    \end{itemize}
    \item \textbf{Variable Selection}: Elastic Net regularization selected key predictors
\end{itemize}
\end{frame}

% ============================================
% METHODOLOGY
% ============================================
\begin{frame}{Methodology}
\textbf{Box-Jenkins Approach}:
\begin{itemize}
    \item Grid search: AR($p$), MA($q$), ARIMA($p,0,q$)
    \item Order selection: AIC/BIC (in-sample)
    \item Model type selection: Out-of-sample RMSE
    \item Diagnostic checking: Ljung-Box, residual analysis
\end{itemize}

\vspace{0.3cm}
\textbf{Volatility Models}:
\begin{itemize}
    \item GARCH($p,q$) vs ARCH($q$)
    \item Selection: BIC (captures volatility dynamics)
    \item Captures volatility clustering and persistence
\end{itemize}

\vspace{0.3cm}
\textbf{Evaluation}: Rolling-origin backtesting (80/20 split, hundreds of test folds)
\end{frame}

% ============================================
% RESULTS
% ============================================
\begin{frame}{Model Selection Results}
\begin{columns}
\column{0.5\textwidth}
\textbf{Return Forecasting}
\begin{table}
\centering
\small
\begin{tabular}{lcc}
\toprule
Model & RMSE & Dir. Acc. \\
\midrule
\textbf{ARIMA(2,0,2)} & \textbf{Lowest} & Lowest \\
AR(8) & Slightly higher & Higher \\
MA(2) & Slightly higher & \textbf{Highest} \\
\bottomrule
\end{tabular}
\end{table}

\vspace{0.2cm}
\textbf{Selected: ARIMA(2,0,2)}
\begin{itemize}
    \item Lowest RMSE (primary criterion)
    \item Magnitude accuracy $>$ directional accuracy
\end{itemize}

\column{0.5\textwidth}
\textbf{Volatility Forecasting}
\begin{table}
\centering
\small
\begin{tabular}{lc}
\toprule
Model & BIC \\
\midrule
\textbf{GARCH(1,1)} & \textbf{Lowest} \\
ARCH(1) & Higher \\
\bottomrule
\end{tabular}
\end{table}

\vspace{0.2cm}
\textbf{Selected: GARCH(1,1)}
\begin{itemize}
    \item Lower BIC
    \item Time-varying volatility
    \item Captures clustering
\end{itemize}
\end{columns}
\end{frame}

\begin{frame}{Forecasts}
\begin{columns}
\column{0.5\textwidth}
\textbf{Return Forecasts (21 days)}
\begin{itemize}
    \item Positive cumulative return
    \item Large majority positive days
    \item Bullish short-term outlook
    \item Wide prediction intervals
\end{itemize}

\column{0.5\textwidth}
\textbf{Volatility Forecasts}
\begin{itemize}
    \item Above historical mean
    \item Time-varying (increasing trend)
    \item Elevated risk ahead
    \item Risk management implications
\end{itemize}
\end{columns}

\vspace{0.3cm}
\begin{center}
\includegraphics[width=0.6\textwidth]{outputs/sp500_forecast.png}
\end{center}
\end{frame}

% ============================================
% CONCLUSIONS
% ============================================
\begin{frame}{Conclusions}
\textbf{Key Findings}:
\begin{enumerate}
    \item \textbf{ARIMA(2,0,2)} best for returns (lowest RMSE)
    \item \textbf{GARCH(1,1)} best for volatility (realistic time-varying forecasts)
    \item \textbf{Bullish outlook} but \textbf{increasing volatility} - balance opportunity with risk
\end{enumerate}

\vspace{0.3cm}
\textbf{Practical Implications}:
\begin{itemize}
    \item Investment: Maintain equity exposure, but account for uncertainty
    \item Risk: Reduce position sizes, increase hedging, adjust VaR upward
\end{itemize}

\vspace{0.3cm}
\textbf{Note}: Forecasts are probabilistic. Models assume stationarity and may break down during regime changes.
\end{frame}

\begin{frame}{Thank You}
\begin{center}
\Large Thank you for your attention!

\end{center}
\end{frame}

\end{document}
